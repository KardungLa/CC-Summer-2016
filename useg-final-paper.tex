%++++++++++++++++++++++++++++++++++++++++
% Don't modify this section unless you know what you're doing!
\documentclass[a4paper,12pt]{article}
\usepackage{tabularx} % extra features for tabular environment
\usepackage{amsmath}  % improve math presentation
\usepackage{graphicx} % takes care of graphic including machinery
\usepackage[margin=1in,a4paper]{geometry} % decreases margins
\usepackage{cite} % takes care of citations
\usepackage[final]{hyperref} % adds hyper links inside the generated pdf file
\hypersetup{
	colorlinks=true,       % false: boxed links; true: colored links
	linkcolor=blue,        % color of internal links
	citecolor=blue,        % color of links to bibliography
	filecolor=magenta,     % color of file links
	urlcolor=blue         
}
%++++++++++++++++++++++++++++++++++++++++


\begin{document}

\title{Introduction to Compiler Construction 2016}
\author{Alexander Mayer (1321354), Daniel Schlager (1320799)}
\date{\today}
\maketitle

\begin{abstract}
In this class we had to implement 11 features to extend the existing selfie-project.
\end{abstract}

\section{Assignment 0}
Team: USEG\\
Members: Alexander Mayer (1321354), Daniel Schlager (1320799)

\section{Assignment 1: Bitwise Shift Instructions}

\section{Assignment 2: Bitwise Shift Operators (Scanning, Parsing)}

\section{Assignment 3: Bitwise Shift Operators (Code Generation, Self-Compilation)}

\section{Assignment 4: Constant Folding}

\section{Assignment 5: Arrays}

\section{Assignment 6: Two-Dimensional Arrays}

\section{Assignment 7: Struct Declarations}

\section{Assignment 8: Struct Access}

\section{Assignment 9: Boolean Operators (Individual)}

\section{Assignment 10: Boolean Operators (Algebra)}

\section{Assignment 11: Memory Management}

\begin{thebibliography}{99}

\bibitem{cksystemsteaching}
C. Kirsch, \textit{Selfie Project},
\texttt{https://github.com/cksystemsteaching/CC-Summer-2016/tree/selfie-master}

\end{thebibliography}


\end{document}
